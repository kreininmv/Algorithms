\documentclass[a4paper,14pt]{article} % тип документа
%\documentclass[14pt]{extreport}
\usepackage{extsizes} % Возможность сделать 14-й шрифт


\usepackage{geometry} % Простой способ задавать поля
\geometry{top=25mm}
\geometry{bottom=35mm}
\geometry{left=20mm}
\geometry{right=20mm}

\setcounter{section}{0}

%%%Библиотеки
%\usepackage[warn]{mathtext}
%\usepackage[T2A]{fontenc} % кодировка
\usepackage[utf8]{inputenc} % кодировка исходного текста
\usepackage[english,russian]{babel} % локализация и переносы
\usepackage{caption}
\usepackage{listings}
\usepackage{amsmath,amsfonts,amssymb,amsthm,mathtools}
\usepackage{wasysym}
\usepackage{graphicx}%Вставка картинок правильная
\usepackage{float}%"Плавающие" картинки
\usepackage{wrapfig}%Обтекание фигур (таблиц, картинок и прочего)
\usepackage{fancyhdr} %загрузим пакет
\usepackage{lscape}
\usepackage{xcolor}
\usepackage{dsfont}
%\usepackage{indentfirst}
\usepackage[normalem]{ulem}
\usepackage{hyperref}




%%% DRAGON STUFF
\usepackage{scalerel}
\usepackage{mathtools}

\DeclareMathOperator*{\myint}{\ThisStyle{\rotatebox{25}{$\SavedStyle\!\int\!\!\!$}}}

\DeclareMathOperator*{\myoint}{\ThisStyle{\rotatebox{25}{$\SavedStyle\!\oint\!\!\!$}}}

\usepackage{scalerel}
\usepackage{graphicx}
%%% END 

%%%Конец библиотек

%%%Настройка ссылок
\hypersetup
{
colorlinks=true,
linkcolor=blue,
filecolor=magenta,
urlcolor=blue
}
%%%Конец настройки ссылок


%%%Настройка колонтитулы
	\pagestyle{fancy}
	\fancyhead{}
	\fancyhead[L]{Домашнее задание}
	\fancyhead[R]{Крейнин Матвей, группа Б05-005}
	\fancyfoot{}
    \fancyfoot[C]{\thepage}
    \fancyfoot[R]{Основные алгоритмы}
%%%конец настройки колонтитулы



\begin{document}
%%%%Начало документа%%%%

\section{Задание 2}

\subsection{Задача 1}
\textbf{а)} $238 \cdot x + 385 \cdot y = 133$

Будем искать НОД(238, 235), используя алгоритм Евклида.
\begin{equation*}
gcd(238, 385) = gcd(147, 238) = gcd(91, 147) = gcd(56, 91) = gcd(35, 56) = \rightarrowtail
\end{equation*}

\begin{equation*}
	= gcd(21, 35) = gcd(14, 21) = gcd(7, 14) = gcd(0, 7) = 7
\end{equation*}
НОД(385, 238)$ = 7$. $133 / 7 = 19$, поэтому решения будут.

Теперь используем расширенный алгоритм Евклида:

\begin{table}[H]
	\begin{tabular}{|l|l|l|}
	\hline
	x   & y  & 238x+385y \\ \hline
	0   & 1  & 385       \\ \hline
	1   & 0  & 238       \\ \hline
	-1  & 1  & 147       \\ \hline
	2   & -1 & 91        \\ \hline
	-3  & 2  & 56        \\ \hline
	5   & -3 & 35        \\ \hline
	-8  & 5  & 21        \\ \hline
	13  & -8 & 14        \\ \hline
	-21 & 13 & 7         \\ \hline
	\end{tabular}
\end{table}

Тогда решениями будут $x = -21 \cdot \frac{133}{7}, y = 13 \cdot \frac{133}{7}$; $x = -399, y = 247$


$x = -399 - k \cdot \frac{385}{\text{НОД(238, 385)}}, y = 247 + k \cdot \frac{238}{\text{НОД(238, 385)}}, k \in \mathds{Z}$


\underline{\textbf{Ответ:}} $x = -399 - 55 \cdot k, y = 247 + 34 \cdot k, k \in \mathds{Z}$
\newline
\textbf{б)} $143 \cdot x + 121 \cdot y = 52$

Будем искать НОД(143, 21): 
\begin{equation*}
gcd(143, 121) = gcd(121, 22) = gcd(22, 11) = gcd(11, 0) = 11
\end{equation*}
Но 52 не делится на 11, поэтому решений нет.


\underline{\textbf{Ответ:}} $\emptyset$


\subsection{Задача 2}
$68x + 85 = 0 mod 561$, или же если переписать $561 \cdot y - 65 \cdot x = 85$

Используем алгоритм Евклида, чтобы найти НОД(561, 68): $gcd(561, 68) = gcd(68, 17) = gcd(17, 0)$

НОД(561, 68) $= 17$,  причем 85 mod $17 = 0$, значит решения есть.

Используем расширенный алгоритм Евклида.

\begin{table}[H]
	\begin{tabular}{|l|l|l|}
	\hline
	x  & y & 238x+385y \\ \hline
	0  & 1 & 385       \\ \hline
	1  & 0 & 238       \\ \hline
	-1 & 1 & 147       \\ \hline
	\end{tabular}
\end{table}

Т.е. решения: $x = 8 \cdot \frac{85}{17} = 40, y = 1 \cdot \frac{85}{17} = 5$


$x = 40 - k \cdot \frac{561}{\text{НОД(561, 68)}} = 40 - 33 \cdot k, y = 5 + k \cdot \frac{-68}{\text{НОД(561, 68)}} = 5 - 4 \cdot k, k \in \mathds{Z}$

\underline{\textbf{Ответ:}} $x = 40 - 33 \cdot k, y = 5 - 4 \cdot k, k \in [0; 16]$

\subsection{Задача 3}
Вычислить $7^{13} mod 167$, используя алгоритм быстрого возведения в степень.
Условимся, что '$=$' обозначает по остатку 167 и каждый раз писать mod 167 я не буду.

\begin{equation*}
	7^{13} = 7 \cdot (7^6)^2 = 7 \cdot ((7^3)^2)^2 = 7 \cdot ((7 \cdot (7 \cdot 7))^2)^2 = 7 \cdot ((9)^2)^2 = 7 \cdot 48 = 2	
\end{equation*}


\underline{\textbf{Ответ:}} 2.

\subsection{Задача 4}


\underline{\textbf{Ответ:}}

\subsection{Задача 5}
\textbf{1)} Найти асимтотику роста функции $T_1(n) = T_1(n-1) + cn$, (при $n > 3$)
\begin{equation*}
	T_1(n) = T_1(n-1) + c \cdot n = T_1(n-2) + c \cdot (n - 1) + c \cdot n = ... = T_1(3) + c \cdot (4 + ... + n)
\end{equation*}

\begin{equation*}
	T_1(n) = 1 + c \cdot (\frac{1+n}{2} \cdot n - 6)
\end{equation*}

Следовательно $T_1(n) = \Theta(n^2)$
\newline
\textbf{2)}
Из курса Дискретного анализа можем записать хар. многочлен: $\lambda^3 = \lambda^2 + 4$, теперь угадаем корень $\lambda_1 = 2$

$(\lambda -2) \cdot (\lambda^2 + \lambda +2) = 0$ $\rightarrow$ $\lambda = \{ \frac{-1 + \pm \sqrt{7}i}{2}$

Тогда получим: 
\begin{equation*}
	T_2(n) = A \cdot 2^n + B \cdot ( (\frac{-1-7\sqrt{7}i}{2})^n + (\frac{-1-7\sqrt{7}i}{2})^n))
\end{equation*}

\begin{equation*}
	T_2(n) = A \cdot 2^n + B \cdot 2^{n/2} \cdot (\frac{-1-7\sqrt{7}i}{2 \sqrt{2}})^n + \frac{-1+7\sqrt{7}i}{2 \sqrt{2}})^n) 	
\end{equation*}


Следовательно $log(T_2(n)) = \Theta(n)$ ч.т.д.
\newline
\textbf{3)} Из прошлого пункта получим ответ.

\underline{\textbf{Ответ:}}
$T_2(n) = A \cdot 2^n + B \cdot 2^{n/2}$


\end{document}