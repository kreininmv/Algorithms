\documentclass[a4paper,14pt]{article} % тип документа
%\documentclass[14pt]{extreport}
\usepackage{extsizes} % Возможность сделать 14-й шрифт


\usepackage{geometry} % Простой способ задавать поля
\geometry{top=25mm}
\geometry{bottom=35mm}
\geometry{left=20mm}
\geometry{right=20mm}

\setcounter{section}{0}

%%%Библиотеки
%\usepackage[warn]{mathtext}
%\usepackage[T2A]{fontenc} % кодировка
\usepackage[utf8]{inputenc} % кодировка исходного текста
\usepackage[english,russian]{babel} % локализация и переносы
\usepackage{caption}
\usepackage{listings}
\usepackage{amsmath,amsfonts,amssymb,amsthm,mathtools}
\usepackage{wasysym}
\usepackage{graphicx}%Вставка картинок правильная
\usepackage{float}%"Плавающие" картинки
\usepackage{wrapfig}%Обтекание фигур (таблиц, картинок и прочего)
\usepackage{fancyhdr} %загрузим пакет
\usepackage{lscape}
\usepackage{xcolor}
\usepackage{dsfont}
%\usepackage{indentfirst}
\usepackage[normalem]{ulem}
\usepackage{hyperref}




%%% DRAGON STUFF
\usepackage{scalerel}
\usepackage{mathtools}

\DeclareMathOperator*{\myint}{\ThisStyle{\rotatebox{25}{$\SavedStyle\!\int\!\!\!$}}}

\DeclareMathOperator*{\myoint}{\ThisStyle{\rotatebox{25}{$\SavedStyle\!\oint\!\!\!$}}}

\usepackage{scalerel}
\usepackage{graphicx}
%%% END 

%%%Конец библиотек

%%%Настройка ссылок
\hypersetup
{
colorlinks=true,
linkcolor=blue,
filecolor=magenta,
urlcolor=blue
}
%%%Конец настройки ссылок


%%%Настройка колонтитулы
	\pagestyle{fancy}
	\fancyhead{}
	\fancyhead[L]{Домашнее задание}
	\fancyhead[R]{Крейнин Матвей, группа Б05-005}
	\fancyfoot{}
    \fancyfoot[C]{\thepage}
    \fancyfoot[R]{Основные алгоритмы}
%%%конец настройки колонтитулы



\begin{document}
%%%%Начало документа%%%%

\section{Задание 6}

\subsection{Задача 1}
\begin{equation*}
	\mathds{E}(X) =  (\frac{5}{6})^3 (-100) 
	+ 3  \frac{1}{6} (\frac{5}{6} )^2 \cdot 100
	+ 3  ( \frac{1}{6} )^2 \frac{5}{6} \cdot (200) +
	+ ( \frac{1}{6} )^3  300
\end{equation*}


Первое это не выпало нужное число, второе это выпало только один раз нужное число, второе это выпало два раза нужное число, и третье выпало три раза нужное чиисло.
Перемножать вероятности можем, т.к. то что выпало на каждом отдельном кубике это независимое событие.

\underline{\textbf{Ответ: }} $\mathds{E}(X) = \frac{-1700}{6^3} = -7\frac{47}{54}$.

\subsection{Задача 2}
Пусть в лотерее участвовало всего n человек, тогда 
\newline $M = 0.4 \cdot 100 \cdot n$ -- максимальный выигрыш.
Пусть y человек выиграло в лотерею 5 тысяч или больше. $T = 5000 \cdot y$ -- количество денег, которые выиграли эти люди.
\newline Но по условия задачи T не превышает M, запишем это. 
\newline $T = 5000 \cdot y \leq M = 0.4 \cdot 100 \cdot n$, учитывая, что $p = \frac{y}{n}$ -- вероятность выиграть 5 тысяч или больше.

Получим, что $\frac{y}{n} = p \leq \frac{0.4 * 100}{5000} = 0.008 \leq 10^{-3}$

\underline{\textbf{Доказано}}

\subsection{Задача 3}
Будем смотреть на слово длины 20, у которого будет выделенное вхождение подслова ab. Это подслово мы можем расположить на 19 местах в нашем слове.
После этого мы можем дописать 18 букв с разных сторон. Т.е. получим всего: $19 \cdot 2^{18}$ вариантов.
\newline
Теперь поделим на количество всех слов длины 20 ($2^{20})$, и получим 
\newline
$\mathds{E}(X) = \frac{19 \cdot 2^{18}}{2^{20}} = 19 / 4 = 4.75$. 

\underline{\textbf{Ответ:}} $\mathds{E}(X) = 4.75$
\newpage
\subsection{Задача 4}
Всего у нас $n!$ перестановок, тогда вероятность каждой отдельной перестановки $\mathds{P} = \frac{1}{n!}$.
Рассмотрим некоторую перестановку и перестановку обратную ей, найдем количество инверсий в этих двух перестановках.
Будем рассматривать все пары $1 \leq i < j \leq n$, всего их будет $\frac{n(n-1)}{2}$. Эти пары будут образовывать инверсию либо в перестановке, либо в перевернутой ей.
\newline
(Если j стоит раньше i в обычной перестановке, то j будет стоять после i и уже не будет давать инверсию в обратной перестановке. Аналогично, если j стоит раньше i в обратной перестановке.)
Всего же таких пар из обычной и перевернутой перестановки будет: $\frac{n!}{2}$.
Получаем: $\mathds{E}(X) = \frac{1}{n!} \cdot \frac{n!}{2} \cdot \frac{n(n-1)}{2} = \frac{n(n-1)}{4}$

\underline{\textbf{Ответ: }} $\mathds{E}(X) = \frac{n(n-1)}{4}$


\subsection{Задача 5}
Введём индикатор события $\mathds{1}_i$ $x_i = i$, он будет принимать значение 1, если входящая перестановка будет удовлетворять условию, в остальных случаях 0.
Тогда вероятность такого события будет $p_k = \frac{(n-1)!}{n!} = \frac{1}{n}$. 
\newline
Таким образом, $X = \sum\limits_{i = 1}^n \mathds{1}_i$.
\begin{equation*}
	\mathds{E}(X) = \sum\limits_{i = 1}^n \mathds{E}(\mathds{1}_i) = n \cdot \frac{1}{n} = 1
\end{equation*}

\underline{\textbf{Ответ:}} $\mathds{E}(X) = 1$

\subsection{Задача 6}
Предположим, что это не так и $\mathds{P}[X \geq 6] \geq \frac{1}{10}$.
\newline
Тогда, $\mathds{E}[2^X] = 5 \geq \frac{1}{10} \cdot 2^6 + Y = 6.4 + Y$, но Y неотрицательная, т.к. отрицательных вероятностей у нас нет, а $2^X$ -- положительная функция.
\newline 
Следовательно наше предположение неверное и $\mathds{P}[X \geq 6] < \frac{1}{10}.$

\underline{\textbf{Доказано}}

\subsection{Задача 7}
Пусть у нас будет такой алгоритм <<выбирания>> независимого множества.
Мы берем веришну и включаем её в множество, а все остальные можем включить тогда, когда их включение приводит к тому, что множество остаётся независимым, 
то есть эта включимая вершина будет не соединена ни с одной вершиной, которая уже есть в этом множестве.
\newline
Пусть $X = X_1 + ... + X_n$, где $X_i$ - это либо включена или не включена i-ая вершина в независимое множество.
\newline
Рассмотрим i-вершину, у которой есть k ребёр, у которых будут номера $i_1, ..., i_k$.
Очевидно, что если она будет стоять на первом месте, то она точно войдёт, вероятность этого будет $\frac{1}{k+1}$.
\newline
Т.е. нужно найти $\mathds{E}(X) = \sum\limits_{k = 1}^n \mathds{P}_k \geq \sum\limits_{k = 1}^n \frac{1}{deg(k)+1} $
\newline 
Теперь вспоним о неравенстве для средних, т.е.
\begin{equation*}
	\frac{\frac{1}{(deg(k_1) + 1)} + ... +\frac{1}{(deg(k_n) + 1)}}{n} \geq \frac{n}{(deg(k_1) +1) + ... + (deg(k_n) +1)}
\end{equation*}
Получаем, что 
\begin{equation*}
\mathds{E}(X) \geq \frac{\frac{1}{(deg(k_1) + 1)} + ... +\frac{1}{(deg(k_n) + 1)}}{n} \geq \frac{n}{(deg(k_1) +1) + ... + (deg(k_n) +1)} \geq \frac{n}{n + nd/2} 
\end{equation*}

\begin{equation*}
	\mathds{E}(X) \geq \frac{n}{n + nd/2} = \frac{2n}{d + 2} \geq \frac{n}{2d}
\end{equation*}

Вот мы и получили, что у нас есть независимое множество размера $\frac{n}{2d}$.

\underline{\textbf{Доказано}}

\end{document} 