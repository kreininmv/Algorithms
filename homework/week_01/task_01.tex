\documentclass[a4paper,14pt]{article} % тип документа
%\documentclass[14pt]{extreport}
\usepackage{extsizes} % Возможность сделать 14-й шрифт


\usepackage{geometry} % Простой способ задавать поля
\geometry{top=25mm}
\geometry{bottom=35mm}
\geometry{left=20mm}
\geometry{right=20mm}

\setcounter{section}{0}

%%%Библиотеки
%\usepackage[warn]{mathtext}
%\usepackage[T2A]{fontenc} % кодировка
\usepackage[utf8]{inputenc} % кодировка исходного текста
\usepackage[english,russian]{babel} % локализация и переносы
\usepackage{caption}
\usepackage{listings}
\usepackage{amsmath,amsfonts,amssymb,amsthm,mathtools}
\usepackage{wasysym}
\usepackage{graphicx}%Вставка картинок правильная
\usepackage{float}%"Плавающие" картинки
\usepackage{wrapfig}%Обтекание фигур (таблиц, картинок и прочего)
\usepackage{fancyhdr} %загрузим пакет
\usepackage{lscape}
\usepackage{xcolor}
\usepackage{dsfont}
%\usepackage{indentfirst}
\usepackage[normalem]{ulem}
\usepackage{hyperref}




%%% DRAGON STUFF
\usepackage{scalerel}
\usepackage{mathtools}

\DeclareMathOperator*{\myint}{\ThisStyle{\rotatebox{25}{$\SavedStyle\!\int\!\!\!$}}}

\DeclareMathOperator*{\myoint}{\ThisStyle{\rotatebox{25}{$\SavedStyle\!\oint\!\!\!$}}}

\usepackage{scalerel}
\usepackage{graphicx}
%%% END 

%%%Конец библиотек

%%%Настройка ссылок
\hypersetup
{
colorlinks=true,
linkcolor=blue,
filecolor=magenta,
urlcolor=blue
}
%%%Конец настройки ссылок


%%%Настройка колонтитулы
	\pagestyle{fancy}
	\fancyhead{}
	\fancyhead[L]{Домашнее задание}
	\fancyhead[R]{Крейнин Матвей, группа Б05-005}
	\fancyfoot{}
    \fancyfoot[C]{\thepage}
    \fancyfoot[R]{ТРЯП}
%%%конец настройки колонтитулы



\begin{document}
%%%%Начало документа%%%%

\section{Задание 1}
\subsection{Задача 1}
\subsubsection*{а)}
Алгоритм выведет все простые числа от 2 до n, если в k-ой ячейке был 0, то номер в этой ячейке не делился ни на какие числа, меньшие его. Значит это простое число.

\underline{\textbf{Ответ:}} Простые числа от 2 до n.

\subsubsection*{б)} N раз алгоритм проходит с k-ой позиции до конца с шагом 1, где $1 \leq k \leq n$, сл-но асимптотика будет f(n) $= O(n^2$).

\underline{\textbf{Ответ:}} f(n)$ = Om(n^2)$.

\subsubsection*{в)} В прошлом пункте было показано, что f(n)$ = O(n^2)$

\underline{\textbf{Ответ:}} Да, является.

\subsection{Задача 2}
$g(n) = \Theta (f(n)) \Leftrightarrow \exists C_1, C_2 > 0, N \in \mathbb{N} : \forall n > N \quad C_1 \cdot f(n) \leqslant g(n) \leqslant C_2 \cdot f(n) $

\subsubsection*{a)} $c < 1$	
\newline
Сумма бесконечно малой геометрической прогрессии: g(n) $= \frac{1}{1-c}$, 
т.е. $g(n) = \theta(1)$

\subsubsection*{b)} $c = 1$
\newline	
$g(n) = n + 1$, возьмём $C_1 = 1$ и $C_2 = 2$, т.е. $g(n) = \Theta(n)$

\subsubsection*{c)} $c > 1$
\newline
Сумма геом. прогрессии: $g(n) = \frac{c^n - 1}{c-1} = \frac{1}{c-1} c^n - \frac{1}{c-1} = \Theta(c^n)$, т.е. $g(n) = \Theta(c^n)$

\subsection{Задача 3}
\subsubsection*{а)} 
$g(n) = O(f(n)) \leftrightarrow \exists C > 0, N \in \mathds{N} : \forall n > N$ $g(n) \leq C \cdot f(n)$
\newline
$\forall N \geq 10 \rightarrow log(n) \geq 1 \rightarrow nlog(n) \geq n \rightarrow n = O(nlog(n))$

\underline{\textbf{Ответ:}} Да, верно.

\subsubsection*{б)}
$g(n) = \Theta(f(n)) \leftrightarrow \exists C_1, C_2 > 0, N \in \mathds{N} : \forall n > N$ $C_1 \cdot f(n) \leq g(n) \leq C_2 \cdot f(n)$
\newline
Возьмём отрицание от условия нижней границы: $\forall c > 0 \forall \Rightarrow \exists n \geq N : nlog(n) < c \cdot n^{1+\varepsilon}$ или же $log(n) < c \cdot n^{\varepsilon}$.
Теперь возьмём производные от обоих частей, получим: $\frac{1}{ln(10)n} < c \cdot \varepsilon n^{\varepsilon - 1}$, $1 < c \cdot \varepsilon n^{\varepsilon}$. Т.к. $\varepsilon > 0$,
то при достаточно больших n это условие будет выполнено. А сл-но производная правой части будет меньше, чем производная левой части. Т.к. обе функции устремляются к бесконечности, то при больших n левая часть будет меньше, чем правая часть. А значит такое n по условиям задачи найдется. 
Т.е. $\not\exists \varepsilon > 0 : nlog(n) = \Omega(n^{1+\varepsilon})$

\underline{\textbf{Ответ:}} Нет, неверно.

\subsection{Задача 4}
\subsubsection*{1.а)}
Возьмём f(n) $= n \cdot log(n)$, а $g(n) = 1$. Тогда $h(n) = n \cdot log(n) = \Theta(n \cdot log(n))$

\underline{\textbf{Ответ:}} Да, может.

\subsubsection*{1.б)} $h(n) = \frac{f(n)}{g(n)} \leq \frac{c_1 \cdot n^2}{c_2 \cdot 1}$
Получается, что $h(n) = O(n^2)$, т.е. $h(n) \not= \Omega(n^3) \rightarrow h(n) \not= \Theta(n^3)$

\underline{\textbf{Ответ:}} Нет, не может.

\subsubsection*{2.)} Как показано ранее $h(n) = O(n^2)$ (например при $f(n) = n^2, g(n) = 1)$.
Т.к. у нас нет нижней оценки на f(n), то она может быть сколько угодной малой, а значит соклько угодно малой может быть и h(n), а значит у нас нет нижней оценки на h(n).
(Если считать, что можно быстрее, чем за O(1), то она конечно есть и равна O(1)).

\underline{\textbf{Ответ:}} $h(n) = O(n^2)$, нижней оценки на h(n) нет.


\subsection{Задача 5}
От внешнего цикла получаем $k_1 = log(n)$, от внутреннего цикла $0 < i < log(n); k_2 = log(n)$, и ещё от двух внутренних циклов $0 \leq j \leq n/2, k_3 = n / 2$ и $1 \leq j \leq n, k_4 = log(n)$.

Итого g(n) $= \Theta(k_1 \cdot k_2 \cdot (k_3 + k_4)) = \Theta(log(n) \cdot log(n) \cdot (n/2 + log(n))) = \Theta(n \cdot log^2(n))$

\underline{\textbf{Ответ:}} $g(n) = \Theta(n \cdot log^2(n))$

\subsection{Задача 7}
Переменные: A, B, C - массивы, соотвественно счетчики x, y, z массивов A, B, C и $N_a, N_b, N_c$ - их размеры. k - количество различных элементов. 
1. Будем считать, что в массивах A, B, C первые элементы идут в порядке неубывания, т.е. $A[0] \leq B[0] \leq C[0]$, если это не так перенумеруем.

2. На этом шаге сравниваем элемент A[0] с B[0] и C[0].

2.1 Если A[0] не совпало ни с кем, то увеличиваем x на один.

2.2 Если совпало A[0] с B[0] и не совпало A[0] с C[0], то $x+=1$, $y +=1$ и $k += 2$.
(Если A[0] совпадет с C[0], то аналогично, но уже увеличиваем z).

2.3 Если совпали все три, то $x += 1, y += 1, z += 1, k += 3$

3. На следующем шаге шаге опять выбираем минимальный элемент из трёх и проделываем эту операцию.

4. Продолжаем до того момента, когда не закончится необработанные элементы в массиве.

\textbf{Корректность:} По причине того, что число элементов конечно, то из них всегда будет наименьший элемент, значит мы посчитаем все не повторяющиеся элементы. 
Почему же мы не посчитаем больше? Всё очень просто, у нас элементы внутри каждого массива различны, поэтому если мы его посчитаем в одном массиве, то в этом же массиве он больше никогда не встретиться, а количество элементов мы увеличивем согласованно. Т.е. сколько встретилось на определенном шаге, столько и получим.
Причем на каждом шаге, мы считаем, что $A[x] \leq B[y] \leq C[z]$, это реализуемо технически т.к. после каждого мы можем переназывать переменные местами, технически это обойдется за константное время, поэтому с этим тоже нет проблем.

\textbf{Оценка:} $ n = N_a + N_b + N_c$
1) По памяти: $N_a + N_b + N_c + 7 = \Theta(n)$ - линейна
2) По времени: $C_1 \cdot N_a \cdot O(1) + C_2 \cdot N_b \cdot O(1) + C_3 \cdot N_c \cdot O(1) = \Theta(n)$ - линейна.


\subsection{Задача 9}
Будем записывать элементы последовательности в стек и если элемент вершины стека не равен элементу,  который мы хотим положить на вершину,  то удаляем элемент из вершины стека,  а тот элемент,  который должны записать, не записываем. 
Тогда в стеке останется только элемент, который встречается в последовательности больше половины раз или же несколько таких элементов тоже может быть.

\textbf{Корректность:} Мы не можем удалить все элементы,  которые встречаются больше половины раз, так как мы обрабатываем парами, в которых элементы не равны.
То есть хотя бы 2 элемента останутся (мы удаляем элементы, не равные искомому, вместе с искомым).

\textbf{Оценка:} 

по времени: $O(2n) = O(n)$, т.к. каждый элемент можем записать в стек и удалить из стека(хотя такого никогда не будет). Заметим, что O(n) возможно, когда все элементы равны.
Т.е. O(n).

по памяти: O(n) - всего элементов n,  их все можно записать в массив, когда все элементы равны между собой.
\end{document}