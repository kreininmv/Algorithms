\documentclass[a4paper,14pt]{article} % тип документа
%\documentclass[14pt]{extreport}
\usepackage{extsizes} % Возможность сделать 14-й шрифт


\usepackage{geometry} % Простой способ задавать поля
\geometry{top=25mm}
\geometry{bottom=35mm}
\geometry{left=20mm}
\geometry{right=20mm}

\setcounter{section}{0}

%%%Библиотеки
%\usepackage[warn]{mathtext}
%\usepackage[T2A]{fontenc} % кодировка
\usepackage[utf8]{inputenc} % кодировка исходного текста
\usepackage[english,russian]{babel} % локализация и переносы
\usepackage{caption}
\usepackage{listings}
\usepackage{amsmath,amsfonts,amssymb,amsthm,mathtools}
\usepackage{wasysym}
\usepackage{graphicx}%Вставка картинок правильная
\usepackage{float}%"Плавающие" картинки
\usepackage{wrapfig}%Обтекание фигур (таблиц, картинок и прочего)
\usepackage{fancyhdr} %загрузим пакет
\usepackage{lscape}
\usepackage{xcolor}
\usepackage{dsfont}
%\usepackage{indentfirst}
\usepackage[normalem]{ulem}
\usepackage{hyperref}




%%% DRAGON STUFF
\usepackage{scalerel}
\usepackage{mathtools}

\DeclareMathOperator*{\myint}{\ThisStyle{\rotatebox{25}{$\SavedStyle\!\int\!\!\!$}}}

\DeclareMathOperator*{\myoint}{\ThisStyle{\rotatebox{25}{$\SavedStyle\!\oint\!\!\!$}}}

\usepackage{scalerel}
\usepackage{graphicx}
%%% END 

%%%Конец библиотек

%%%Настройка ссылок
\hypersetup
{
colorlinks=true,
linkcolor=blue,
filecolor=magenta,
urlcolor=blue
}
%%%Конец настройки ссылок


%%%Настройка колонтитулы
	\pagestyle{fancy}
	\fancyhead{}
	\fancyhead[L]{Домашнее задание}
	\fancyhead[R]{Крейнин Матвей, группа Б05-005}
	\fancyfoot{}
    \fancyfoot[C]{\thepage}
    \fancyfoot[R]{Основные алгоритмы}
%%%конец настройки колонтитулы



\begin{document}
%%%%Начало документа%%%%

\section{Задание 11}
\subsection{Задача 1}

а) После добавления ко всем ребрам $w$ вес любого остовного дерева станет $s + (n - 1)w$,  где $s$ - вес этого дерева до модификации.  
(В каждом остовном дереве $n$ вершин и $n - 1$ ребро.  Поэтому после модификации минимальное остовное дерево останется минимальным,  так как ко всем остовным деревьям прибавили одну и ту же константу.

\textbf{Ответ:} правда
\newline
б) Допустим противное и минимальное ребро $а$ не входит в минимальное остовное дерево.  
Тогда добавим его - получим цикл,  в котором есть это ребро $a$,  выкинув любое другое ребро из этого цикла,  получим остовное дерево меньшего веса - противоречие.  

\textbf{Ответ:} правда
\newline
в) Допустим противное,  ребро $e$ не является самым легким ребром пересекающий некоторый разрез,  тогда выкинем его и добавим ребро минимального веса,  пересекающее этот разрез,  тогда получится остовное дерево меньшего веса - противоречие.  
(Как будет выглядеть это на картинке: удалили ребро из исходного дерева - стало две компоненты связанности - взяли ребро наименьшего веса,  соединяющее их - получили остовное дерево меньшего веса.)

\textbf{Ответ:} правда
\newline
г) Приведем контр-пример,  рассмотрим граф на 3 вершинах: $|AB| = 2,  |AC| = 2,  |BC| = 3$.  Тогда минимальное остовное дерево будет: $B - A - C$.  Кратчайший путь между $B$ и $C$ - ребро $|BC|$ и оно не лежит в минимальном остовном дереве.

\textbf{Ответ:} неправда
 

\subsection*{Задача 2}
 
Воспользуемся алгоритмом Крускала для графа $G$ - получим минимальное остовное дерево для графа $G$.  Потом воспользуемся алгоритмом Крускала для подграфа $H$,  причем отсортируем ребра так,  чтобы ребра в отсортированном массиве образовывали подпоследовательность отсортированного массива (нужно для того,  чтобы мы в массиве для $H$ брали ребра(которые есть) в той же последовательности,  как и в $G$).  Заметим,  что если ребро $e$ было добавлено в первом случае,  то оно будет добавлено и во втором.  Допустим противное,  в первом случае оно добавлено,  а во втором нет.  Это означает,  что во втором случае мы пытаемся соединить вершины из одной компоненты связанности,  но это значит,  что и в первом случае мы бы пытались соединить вершины из одной компоненты связанности,  так как в первом случае есть те же самые ребра до $e$,  добавляющие эти вершины в одну компоненту связанности - противоречие.  Значит ребра,  входящие,  как в $T$,  так и в $H$,  входят в некоторое минимальное остовное дерево подграфа $H$.
(Заметим,  что все минимальные остовные деревья графа $H$ получаются перестановкой ребер одинакового веса в отсортированном массиве ребер - значит наше доказательство верно для любого остовного дерева графа $H$.)
 
\subsection{Задача 3}
 
Разделим множества на пары: четная вершина и нечетная - объединим их,  так, чтобы нечетная вершина была корнем.  Опять разделим на пары и т. д. (каждый раз подаем в $Union$ корни множеств.  Получим бинарное дерево,  тогда время затраченное на объединение: $\frac{n}{2} + \frac{n}{4} + ...  + \frac{n}{2^{log_2n}} = \Theta(n)$.  Затем для четных элементов будем вызывать $Find$.  Так как они лежат в листьях,  то время каждого запроса $\Theta(logn)$,  а время всех запросов $\Theta((m-n)logn)$.  Тогда суммарное время работы $\Theta(n) + \Theta((m-n)logn) = \Theta(n + (m-n)logn) = \Theta(mlogn) = \Omega(mlogn)$,  так как $m > n$.
  
\subsection{Задача 4}

(Мой алгоритм не обрабатывает граф на 2 вершинах,  но  если он был связанным, то остовным деревом он будет сам,  если не был связным, то таких деревьев нет.)
Заведем двумерный массив размера $V$ на 2.  Если $i$  вершина принадлежит $U$,  то в массив на $i$ месте первой строчки записываем 1,  иначе 0,  вторую строчку заполняем нулями.  Запускаем алгоритм Крускала,  но с некоторым уточнением: каждый раз когда рассматриваем ребро,  смотрим:

\begin{enumerate}
\item Если две вершины принадлежат $U$,  то переходим к следующему ребру;
\item Если только одна вершина принадлежит $U$,  то смотрим на значение во второй строчке для этой вершины,  если там стоит 0,  то пользуемся алгоритмом Крускала и меняем 0 на 1,  если 1,  то переходим к следующему ребру;
\item Если ни одна вершина не принадлежит $U$,  то пользуемся алгоритмом Крускала.
\end{enumerate}

\textbf{Корректность:} следует из корректности алгоритма Крускала,  только мы запретили ребра из $U$ в $U$ и больше одного ребра в вершину из $U$,  понятно,  что надо брать минимальное по весу.

\textbf{Асимптотика:} Время работы алгоритма Крускала: $O(|E|log|V|)$,  а время сравнений $O(|V|)$,  то есть время сравнений $O(|E|log|V|)$.

\end{document} 
