\documentclass[a4paper,14pt]{article} % тип документа
%\documentclass[14pt]{extreport}
\usepackage{extsizes} % Возможность сделать 14-й шрифт


\usepackage{geometry} % Простой способ задавать поля
\geometry{top=25mm}
\geometry{bottom=35mm}
\geometry{left=20mm}
\geometry{right=20mm}

\setcounter{section}{0}

%%%Библиотеки
%\usepackage[warn]{mathtext}
%\usepackage[T2A]{fontenc} % кодировка
\usepackage[utf8]{inputenc} % кодировка исходного текста
\usepackage[english,russian]{babel} % локализация и переносы
\usepackage{caption}
\usepackage{listings}
\usepackage{amsmath,amsfonts,amssymb,amsthm,mathtools}
\usepackage{wasysym}
\usepackage{graphicx}%Вставка картинок правильная
\usepackage{float}%"Плавающие" картинки
\usepackage{wrapfig}%Обтекание фигур (таблиц, картинок и прочего)
\usepackage{fancyhdr} %загрузим пакет
\usepackage{lscape}
\usepackage{xcolor}
\usepackage{dsfont}
%\usepackage{indentfirst}
\usepackage[normalem]{ulem}
\usepackage{hyperref}




%%% DRAGON STUFF
\usepackage{scalerel}
\usepackage{mathtools}

\DeclareMathOperator*{\myint}{\ThisStyle{\rotatebox{25}{$\SavedStyle\!\int\!\!\!$}}}

\DeclareMathOperator*{\myoint}{\ThisStyle{\rotatebox{25}{$\SavedStyle\!\oint\!\!\!$}}}

\usepackage{scalerel}
\usepackage{graphicx}
%%% END 

%%%Конец библиотек

%%%Настройка ссылок
\hypersetup
{
colorlinks=true,
linkcolor=blue,
filecolor=magenta,
urlcolor=blue
}
%%%Конец настройки ссылок


%%%Настройка колонтитулы
	\pagestyle{fancy}
	\fancyhead{}
	\fancyhead[L]{Домашнее задание}
	\fancyhead[R]{Крейнин Матвей, группа Б05-005}
	\fancyfoot{}
    \fancyfoot[C]{\thepage}
    \fancyfoot[R]{Основные алгоритмы}
%%%конец настройки колонтитулы



\begin{document}
%%%%Начало документа%%%%

\section{Задание 3}

\subsection{Задача 1}
Понятно, что на каждой ветке будет $C_1$ операций, и каждая ветка будет порождать ветку от $\frac{n}{4}$, когда $n \leq 2020$, то операций будет $C_2$ (т.к. во втором else цикл от 0 но n, но n не превышает 2020).
$T(n) = 3 \cdot T(\frac{n}{4}) + C$, возьмём за $C = max(C_1, C_2)$, и применим Мастер теорему.

$a = 3, b = 4, d = 0, log_4 3, d = O(n^{log_4 3}),$ тогда это первый случай мастер теоремы и $T(n) = \Theta(n^{log_4 3})$

\underline{\textbf{Ответ:}} $T(n) = \Theta(n^{log_4 3})$

\section{Задача 2}
Кажется, что это алгоритм вычисления НОД для всех чисел массива.
Все числа массива хоть и уменьшаются, но они все остаются положительными, т.к. из большего вычитается меньшее и к тому  же они все различны.
При этом понятно, что все числа в конце концов будут одинаковы, т.к. иначе мы могли из большего вычесть меньшее число. 

Предположим, что $D = \text{НОД}$ -- для всех чисел, логично, что до этого шага были числа $A = a \cdot D, B = b \cdot D$, пусть $A > B$ для определенности.
После же этого шага будет $A = (a - b) \cdot D, B = b \cdot D$. Видно, что числа не могут после каждого шага стать меньше, чем $D$

Пусть выполнение алгоритма завершено и все числа не будут равны $D$, тогда $D \not = \text{НОД}$, приходим к противоречию.

\underline{\textbf{Ответ:}} Наибольший общий делитель всех чисел массива.

\section{Задача 3}
$(a+b)^2 = a^2 + 2\cdot ab + b^2 \longrightarrow ab = \frac{(a+b)^2 - a^2 - b^2}{2}$
\newline
Отсюда видно, что сложение двух чисел за линейку, потом еще возведение в квадрат по предположению тоже за линейку, потом еще две операции сложения тоже за линейку, и деление на два мне было сказано, что тоже за линию выполняется.
Итого получаем, что произведение двух чисел будет производиться за линию, при предположении того, что возводить в квадрат можно за линию.

\underline{\textbf{Доказано}}

\section{Задача 5}
Вспоним о том, что $(\sum_{i=1}^n a_i)^2 = a_1^2+2\cdot a_1 \cdot a_2 + ... + a_n^2$, тогда можем получить следующее выражение:
\begin{equation*}
	\sum_{i \not= j}^n(a_i \cdot a_j) = \frac{(\sum_{i=1}^n a_i)^2 - \sum_{i=1}^n a_i^2 }{2}
\end{equation*}
Получим, что :
\newline
Cложность первого слагаемого будет $O(n)$, т.к. всего n операций сложения и одно возведение в степень.
\newline
Сложность второго слагаемого будет $O(n)$, т.к. всего n операций умножения и n-1 операция сложения.
\newline
И еще одна операция деления, т.е. в итоге получаем $O(n)$

\underline{\textbf{Ответ: }} Получили $O(n)$ от количества операций.

\section{Задача 6}
\subsection{а)}
$T(n) = 36 \cdot T(\frac{n}{6}) + n^2$,   
\newline
$a = 36, b = 6, f(n) = n^2, d = log_b a = 2, f(n) = n^2 =\Theta(n^{2}) = \Theta(n^d)$
\newline
Это будет второй случай мастер теоремы: $T(n) = \Theta(n^2 \cdot logn)$

\underline{\textbf{Ответ:}} $T(n) = \Theta(n^2 logn)$

\subsection{б)}
$T(n) = 3 T(\frac{n}{3}) + n^2$,
\newline
$a = 3, b = 3, f(n) = n^2, d = log_b a = 1, f(n) = n^2 = \Omega(n^{1 + \varepsilon})$.
\newline
$\exists c : 0 < c < 1, a \cdot f(\frac{n}{b}) \leq c \cdot f(n)$
\newline
$3 \cdot \frac{n^2}{9} = \frac{n^2}{3} \leq c \cdot n^2 \longrightarrow c = \frac{1}{2}$, такое $c$ существует и равно $\frac{1}{2}$, сл-но выполняется третий случай Мастер теоремы.

\underline{\textbf{Ответ:}} $T(n) = \Theta(n^2)$

\subsection{в)}
$T(n) = 4 \cdot T(\frac{n}{2}) + \frac{n}{logn},$
\newline
$a = 4, b = 2, d = log_b a = 2, f(n) = \frac{n}{logn}, f(n) = O(n^{2 - \varepsilon})$. 
\newline
Пусть $\varepsilon = \frac{1}{2}$ и это будет выполняться, т.к. любой логарифм
\newline
Это будет первый случай мастер теоремы.

\underline{\textbf{Ответ:}} $T(n) = \Theta(n^2)$

\section{Задача 7}
Воспользуемся сортировкой массива с помощью слияния, когда же будем сливать или же соединять наши два остортированных массива $A_1$ и $A_2$ и будем брать элемент из массива $A_2$, 
то будем прибавлять к счетчику инверсий количество элементов, которые будут стоять в $A_1$ правее этого элемента.

\underline{Корректность:} Предположим, что алгоритм не будет корректным, тогда найдутся инверсии, которые мы не посчитали,
то есть на каком-то шаге m и k, $m < k$ и $A[m] > A[k]$, значит $A[k] \ in A_1$ и $A[m] \in A_2$, т.е. мы их будем учитывать при следующем слиянии. Получили противоречие.

\underline{Асимптотика:} $T(n) = 2 \cdot T(\frac{n}{2}) + C \cdot n,$
\newline
$a = 2, b = 2, d = log_b a = 1, f(n) = C \ cdot = \Theta(n^d)$
\newline
Это будет второй случай мастер теоремы.

\underline{\textbf{Ответ:}} $T(n) = \Theta(n\cdot logn)$

\section{Задача 8}
Доказать, что если $T_1(n) = a \cdot T_1(\frac{n}{b}) + f(n)$, $T_2(n) = a \cdot T_2(\frac{n}{b}) + g(n)$, и $f(n) = \Theta(g(n))$, то
$T_1(n) = \Theta(T_2(n))$

\begin{equation*}
T_1(n) = \sum_{i = 0}^{log_b n} a^i \cdot f(\frac{n}{b^i}) + C_1 \cdot a^{log_b n}
\end{equation*}

\begin{equation*}
T_2(n) = \sum_{i = 0}^{log_b n} a^i \cdot g(\frac{n}{b^i}) + C_2 \cdot a^{log_b n}
\end{equation*}
Учитывая, что $f(n) = \Theta(g(n))$, т.е. $a^i \cdot f(\frac{n}{b^i}) = \Theta(a^i g(\frac{n}{b^i}))$,
\newline
сл-но $T_1(n) = \Theta(T_2(n))$

\underline{Доказано}

\section{Задача 9}
\subsection{а)}
$T(n) = 3 \cdot T(\frac{n}{4}) + T(\frac{n}{6}) + n$
Рассмотрев, несколько строчек рекурсий, можем заметить, что на k-ой строке количество операций будет равно $\frac{11 \cdot}{12}^k \cdot n$
\begin{equation*}
	\sum_{k = 0}^{logn} (\frac{11}{12})^k \cdot n = n \cdot (1 - \frac{11}{12}^{logn})
\end{equation*}
Получаем, что при $\frac{n}{2} \leq T(n) \leq 12 \cdot T(n) \longrightarrow T(n) = \Theta(n)$

\underline{\textbf{Ответ:}} $T(n) = \Theta(n)$

\subsection{б)} 
$T(n) = T(\alpha \cdot n) + T((1- \alpha) \cdot n) + C \cdot n$, $(0 < \alpha 1)$.

Заметим, что на каждом шаге будет $C \cdot n$ операций.
Высота же дерева получится $h = max\{log_{\alpha} n, log_{1- \alpha} n)\}$, видно, что $h = C_2 \cdot logn$, т.к. $log(1-\alpha)$ и $log(\alpha)$ отличаются друг от на константу.
И т.к у нас одинаковое количество операций на каждой ветке, то это не повлияет на итоговый результат. (То есть это сумма ниже будет ограничена снизу и сверху двумя разными константами, умноженными на $n \cdot logn$, поэтому и получается верная асимптотическая оценка).

\begin{equation*}
	\sum_{i = 0}^{C_2 \cdot logn} C \cdot n = C \cdot n \cdot((logn + 1) \cdot C_2)
\end{equation*}

\underline{\textbf{Ответ:}} $T(n) = \Theta(n \cdot logn)$

\subsection{в)}
$T(n) = T(\frac{n}{2}) + 2 \cdot T(\frac{n}{4}) + C \cdot n$

Точно также заметим, что на каждой ветке у нас $C \cdot n$ операций и что высота дерева будет $log_2 n \leq h \leq log_4 n$.
То есть как и в примеры выше, где тоже на каждой ветке было $\Theta(n)$ операций, мы можем сумму снизу ограничить двумя константными, умноженными на $n \cdot logn$.
\begin{equation*}
	\sum_{i = 0}^{logn \cdot c_1} C \cdot n = C \cdot n (c_1(logn + 1))
\end{equation*}

\underline{\textbf{Ответ:}} $T(n) = \Theta(n \cdot logn)$

\subsection{г)}
$T(n) = 27 \cdot T(\frac{n}{3}) + \frac{n^3}{log^2n}$
\newline
Методом долгого вглядывания обнаружим, что $T(n) = \Theta(...)$


\section{Задача 10}
Будем сначала вычислять n!, потом с помощью алгоритма быстрого возведения в степень вычислим $n!^{p - 2} mod p$, и заполним ячейку $invfac[n]$.
После ээтого вычислим $invfac[k]$, где $k \in [1, n-1]$, следующим образом: $invfac[k] = invfac[k+1] \cdot k( mod p)$.

\underline{Корректность:} Поскольку n < p, то n и p будут взаимно просты, следовательно можем воспользоваться малой теоремой Ферма,
то есть $n!^{p-1} \equiv 1 (mod p) \longrightarrow n!^{p-2} \equiv n!^{-1} \equiv (mod p)$.
Для n-ого элемента получили, что значение было вычислено корректно. $n! \cdot n!^{-1} \equiv (n-1)! \cdot (n \cdot n!^{-1}) (mod p \longrightarrow (n-1)!^{-1} \equiv n!^{-1} \cdot n(mod p))$

\underline{Асимптотика:} Для того, чтобы посчитать факториал нам нужно $O(n)$, чтобы быстро возвести  степень $log_2(p)$, каждый оставшийся элементв вычисляется за одну арифметическую операцию, т.е. за $O(n)$.
Получаем, что сложность всего алгоритма $O(n + logp)$

\end{document}