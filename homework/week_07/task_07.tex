\documentclass[a4paper,14pt]{article} % тип документа
%\documentclass[14pt]{extreport}
\usepackage{extsizes} % Возможность сделать 14-й шрифт


\usepackage{geometry} % Простой способ задавать поля
\geometry{top=25mm}
\geometry{bottom=35mm}
\geometry{left=20mm}
\geometry{right=20mm}

\setcounter{section}{0}

%%%Библиотеки
%\usepackage[warn]{mathtext}
%\usepackage[T2A]{fontenc} % кодировка
\usepackage[utf8]{inputenc} % кодировка исходного текста
\usepackage[english,russian]{babel} % локализация и переносы
\usepackage{caption}
\usepackage{listings}
\usepackage{amsmath,amsfonts,amssymb,amsthm,mathtools}
\usepackage{wasysym}
\usepackage{graphicx}%Вставка картинок правильная
\usepackage{float}%"Плавающие" картинки
\usepackage{wrapfig}%Обтекание фигур (таблиц, картинок и прочего)
\usepackage{fancyhdr} %загрузим пакет
\usepackage{lscape}
\usepackage{xcolor}
\usepackage{dsfont}
%\usepackage{indentfirst}
\usepackage[normalem]{ulem}
\usepackage{hyperref}




%%% DRAGON STUFF
\usepackage{scalerel}
\usepackage{mathtools}

\DeclareMathOperator*{\myint}{\ThisStyle{\rotatebox{25}{$\SavedStyle\!\int\!\!\!$}}}

\DeclareMathOperator*{\myoint}{\ThisStyle{\rotatebox{25}{$\SavedStyle\!\oint\!\!\!$}}}

\usepackage{scalerel}
\usepackage{graphicx}
%%% END 

%%%Конец библиотек

%%%Настройка ссылок
\hypersetup
{
colorlinks=true,
linkcolor=blue,
filecolor=magenta,
urlcolor=blue
}
%%%Конец настройки ссылок


%%%Настройка колонтитулы
	\pagestyle{fancy}
	\fancyhead{}
	\fancyhead[L]{Домашнее задание}
	\fancyhead[R]{Крейнин Матвей, группа Б05-005}
	\fancyfoot{}
    \fancyfoot[C]{\thepage}
    \fancyfoot[R]{Основные алгоритмы}
%%%конец настройки колонтитулы



\begin{document}
%%%%Начало документа%%%%

\section{Задание 7}

\subsection{Задача 1}
\begin{equation*}
	e_A = (107, 187); e_B = (7, 253)
\end{equation*}

\begin{equation*}
	\phi_A = \phi(187) = 16 \cdot 10 = 160; \phi_B = \phi(253) = 22 \cdot  10 = 220
\end{equation*}

\begin{equation*}
	d_A = e_A^{-1} \text{ mod } 160; d_B = e_B^{-1} \text{ mod 220}
\end{equation*}

\begin{table}[H]
	\begin{tabular}{|l|l|l|l|l|}
	\hline
	a   & b   & r  & x  & y  \\ \hline
	160 & 107 &    & -2 & 3  \\ \hline
	107 & 53  & 1  & 1  & -2 \\ \hline
	53  & 1   & 2  & 0  & 1  \\ \hline
	1   & 0   & 53 & 1  & 0  \\ \hline
	\end{tabular}
\end{table}

\begin{equation*}
	220 \cdot (-2) + 7 \cdot 63 = 1 \rightarrow d_B = 63
\end{equation*}

\begin{equation*}
	m = 17; m_B = m^{e_B} \text{ mod } 253 = 17^7 \equiv 250 mod 253
\end{equation*}

\begin{equation*}
	17^7 = 17^6 \cdot 17 = 104 \cdot 17 \equiv 250 \text{ mod } 253
\end{equation*}

\begin{equation*}
	17^6 = (17^3)^2 = 106^2 \equiv 104 \text{ mod } 253
\end{equation*}

\begin{equation*}
	17^3 = 17^2 \cdot 17 = 36 \cdot 17 = 106 \text{ mod 253 }
\end{equation*}

\begin{equation*}
	17^2 = 17^2 = 289 = 51 \text{ mod } 253
\end{equation*}

Сертификат: $c = m^{d_A} = 17^3 \equiv 51 mod 187$




\subsection{Задача 2}

\begin{equation*}
	s_y = y^d = r^{ed}x^d \equiv rx^d \text{ mod } n \rightarrow r^{-1}s_y \equiv x^d \text{ mod } n
\end{equation*}

\subsection{Задача 3}
\begin{equation*}
	 \begin{cases}
	   M^3 \equiv_{N_1} C_1\\
	   M^3 \equiv_{N_2} C_2\\
	   M^3 \equiv_{N_3} C_3
	 \end{cases}
\end{equation*}

Используя китайскую теорему об остатках решим эту систему сравнений. Вычислим $N = N_1 N_2 N_3$.
Для каждого $i \in \{1, 2, 3\}$ найдем $K_i = \frac{N}{N_i}$. Расширенным алгоритмом Евклида найдем $K_i^{-1} \text{ mod } N_i$.
Теперь найдем решение $M^3 = \sum\limits_{i = 1}^3 C_iK_iK_i^{-1} \text{ mod } N$. По китайской теореме об остатках, существует ровно один вычет по модулю N,
удовлетворяющий системе. Посколько $M < N_i, M^3 < N$. Тогда полученное решение по модулю N равно $M^3$, а значит, можно вычислить $M = \sqrt[3]{M^3}$, например,
итеративным методом Ньютона.

\textbf{Асимптотика:}
При решении системы использовались арифметические операции и алгоритм Евклида -- полиномиальное время. Каждая итерация метода Ньютона
состоит из арифметических операций, при этом с каждой итерацией количество верно вычисленных цифр удваивается -- получается полинамиальный алгоритм от длины числа.

\subsection{Задача 4}
Сначала определим количество чисел от 2 до $N - 1$, которые подходят на роль d.
Для начала рассмотрим все числа, меньшие $\phi(N)$. Из этих чисел среди принадлежащих $\mathds{Z_{\phi(N)}}$ не подходит, т.к. иначе
\begin{equation*}
	ex \equiv_{\phi(N)} 1 \rightarrow x \equiv_{\phi(N)} e^{-1}
\end{equation*}

Поскольку $x \not = e^{-1}$, x должен быть больше $\phi(N)$. Теперь рассмотрим числа, большие либо равные $\phi(N)$.
Таких чисел всего $pq - pq + p + q - 1 = p + q -1 \ll \phi(N)$. Поэтому с очень большой точностью чисел, удовлетворяющих d, найдется всего одно. 
Теперь подсчитаем матожидание количества попыток по определению:

\begin{equation*}
	\mathds{E}[T] = 1 \frac{1}{n-2} + 2 \cdot \frac{n-3}{n-2}\frac{1}{n-3} + 3 \frac{n-3}{n-2}\frac{n-4}{n-3}\frac{1}{n-4}+...
\end{equation*}

\begin{equation*}
	\mathds{E}[T]= \frac{1}{n-2} + 2\frac{1}{n-2}+3\frac{1}{n-2} +...+ (n-2)\frac{1}{n-2} = \frac{(n-2)(n-1)}{2(n-2)} = \frac{n-1}{2}
\end{equation*}

Теперь подсчитаем матожидание количества попыток при полном переборе. Формула получится такой же, что и при случайном выборе, поэтому матожидание будет таким же $\frac{n-1}{2}$


\subsection{Задача 5}
Докажем индукицей по n. Считаем, что всюду $m \leq n$, т.к. в противном случае алгоритм не завершит работу корректно. 
\begin{enumerate}
	\item База индукции $n = 0, 1$ В первом случае возможно только пустое множество, оно и будет возвращено.
	Во втором случае при $m = 0$ возможно тольео пустое множество, оно и возвращается, при $m = 1$ рекурсивный вызов возвращается пустое множество, поэтому
	единственный элемент случайно выбирается функцикй $Random(0, 1)$
	\item Индукционный переход: пусть алгоритм случайно и равновероятно возвращает множество для некоторого $n = k$. 
	\item Докажем, что это так и для $n = k + 1$. Покажем что вероятности вхождения каждого элемента из $[1, n+1]$ в итоговое множество равны.
	В ходе выполнения алгоритма будет рекурсивный вызов, который по предположению индукции вернет случайное равновероятное множество S размера m из первых n элементов.
	Рассмотрим произвольный элемент, не равный $n + 1$. Тогда вероятность вхождения этого элемента в множество равна сумме вероятностей вхождения этого элемента в S, и вероятности 
	вхождения этого элемента при добавлении в S: $P = \frac{m}{n} + \frac{n-m}{n}\frac{1}{n+1} = \frac{m+1}{n+1}$. Рассмотрим $n+1$-й элемент.
	Он входит в множество, если был выбран элемент, который уже принадлежит S, или если был выбран $n+1$. Поскольку в S m элементов, эта вероятность равна $\frac{m}{n+1} + \frac{1}{n+1} = \frac{m+1}{n+1}$.
	Таким образом, у всех элементов одинаковые вероятности дополнить случайное монжество. Значит полученное при дополнении множество будет случайным.
\end{enumerate}
\end{document} 
