\documentclass[a4paper,14pt]{article} % тип документа
%\documentclass[14pt]{extreport}
\usepackage{extsizes} % Возможность сделать 14-й шрифт


\usepackage{geometry} % Простой способ задавать поля
\geometry{top=25mm}
\geometry{bottom=35mm}
\geometry{left=20mm}
\geometry{right=20mm}

\setcounter{section}{0}

%%%Библиотеки
%\usepackage[warn]{mathtext}
%\usepackage[T2A]{fontenc} % кодировка
\usepackage[utf8]{inputenc} % кодировка исходного текста
\usepackage[english,russian]{babel} % локализация и переносы
\usepackage{caption}
\usepackage{listings}
\usepackage{amsmath,amsfonts,amssymb,amsthm,mathtools}
\usepackage{wasysym}
\usepackage{graphicx}%Вставка картинок правильная
\usepackage{float}%"Плавающие" картинки
\usepackage{wrapfig}%Обтекание фигур (таблиц, картинок и прочего)
\usepackage{fancyhdr} %загрузим пакет
\usepackage{lscape}
\usepackage{xcolor}
\usepackage{dsfont}
%\usepackage{indentfirst}
\usepackage[normalem]{ulem}
\usepackage{hyperref}




%%% DRAGON STUFF
\usepackage{scalerel}
\usepackage{mathtools}

\DeclareMathOperator*{\myint}{\ThisStyle{\rotatebox{25}{$\SavedStyle\!\int\!\!\!$}}}

\DeclareMathOperator*{\myoint}{\ThisStyle{\rotatebox{25}{$\SavedStyle\!\oint\!\!\!$}}}

\usepackage{scalerel}
\usepackage{graphicx}
%%% END 

%%%Конец библиотек

%%%Настройка ссылок
\hypersetup
{
colorlinks=true,
linkcolor=blue,
filecolor=magenta,
urlcolor=blue
}
%%%Конец настройки ссылок


%%%Настройка колонтитулы
	\pagestyle{fancy}
	\fancyhead{}
	\fancyhead[L]{Домашнее задание}
	\fancyhead[R]{Крейнин Матвей, группа Б05-005}
	\fancyfoot{}
    \fancyfoot[C]{\thepage}
    \fancyfoot[R]{Основные алгоритмы}
%%%конец настройки колонтитулы



\begin{document}
%%%%Начало документа%%%%

\section{Задание 4}

\subsection{Задача 1}
Считаем, количество нулей и единиц в массиве, это можно сделать за $O(n)$ операций. Потом первые k элементов заполняю нулями в массиве, следующие $n-k$ элементов единицами.


\subsection{Задача 2}
Упорядочим отрезки, пусть $n+1$ отрезок содержится в n. Тогда нужны интервалы $[l_{\frac{2n}{3}}; l_{\frac{2n}{3}+1})$ и $(r_{\frac{2n}{3}}; r_{\frac{2n}{3}+1}]$. 
Для этого найдём $\frac{2n}{3}$ и $\frac{2n}{3}+1$ порядковые статистики левых граний и $\frac{2n}{3}$ и $\frac{2n}{3}+1$ порядковые статистики правых границ, они находятся за $O(n)$. Получаем, что алгоритм работате за $O(n)$.

\subsection{Задача 3}
Пусть T(n) -- время работы алгоритма, тогда:

\begin{equation*}
	T\left(n\right) \leq T\left(\frac{5n}{7}\right) + T\left(\frac{n}{7}\right) + cn
\end{equation*}
Где $T\left(\frac{n}{7}\right)$ -- время нахождение медианы медиан, $T\left(\frac{5n}{7}\right)$ -- максимальное время работы поиска k порядковой статистики в одной из двух частей массива
(элементов больших медианного элемента и меньших). $\frac{5n}{7}$ -- количество элементов меньших, медианного элемента не меньше, чем $\frac{n}{2} \cdot \frac{4}{7}$, больших столько же $\frac{2n}{7}$. Получаем, что количество элементов меньших, чем медианный элемент не меньше, чем $\frac{2n}{7}$, и не больше, чем $\frac{5n}{7}$. 
Аналогично получаем для элементов больших медианнного значения. Т.к. есть $c \cdot n$, то нижняя оценка, очевидно, это линия. Теперь докажем оценку сверху, пусть есть такое $d : T(n) < d \cdot n$, т.е. $T(n) = O(n)$.
База будет выполняться , т.к. при малых k, $T(k) = \Theta(1)$, т.е. можем выбрать достаточное большое d.
\begin{equation*}
	d \cdot \frac{5n}{7} + d \cdot \frac{n}{7} + c \cdot n \leq d \cdot n
\end{equation*}
Получаем:
\begin{equation*}
	7 \cdot c \leq d
\end{equation*}
Возьмёмум $d = 8 \cdot c$ и это будет верно. Получаем, что $T(n) = \Theta(n)$.

\underline{\textbf{Ответ:}} $T(n) = \Theta(n)$

\subsection{Задача 4}
Будем рассматривать две точки, очевидно, что расстояние от любой точки, лежащей на отрезке, который соединяет эти две точки, будет постоянным и минимальным.
Значит, нужно найти медиану данного массива, т.е. за $O(n)$.

\underline{\textbf{Корректность:}} Рассмотрим такую функцию $Sum(s) = |x_1 - s| + ... + |x_{2n+1} - s|$.
Существует минимум, т.к. Sum(s) всегда будет неотрицательной. Т.к. количество модулей нечётно, то при любом s наклон прямой не равен нулю, получается, что минимумом может быть только конечное количество точек.
Методом пристального взгляда заметим, что медиана данного массива -- это экстремум функции. Т.к. до неё коэффициент наклона прямой был всегда отрицательным, а после неё всегда положительный.

\underline{\textbf{Ответ:}} $\Theta(n)$

\subsection{Задача 5}
Перепишем это уравнение:
\begin{equation*}
	M \cdot y - a \cdot x = b
\end{equation*}
Как можем заметить это диофантовое уравнение, которое можно решить с помощью алгоритма Евклида, который работате за $O(n^3)$.

\underline{\textbf{Ответ: }} Алгоритм Евклида для $M \cdot y - a \cdot x = b$ -- $\Theta(n^3)$. 


\subsection{Задача 6}
Перемножьте многочлены: $f(x) = 2x^3 + 3x^2 + 1$ и $g(x) = 2x^2 + x$ с помощью БПФ. 
\begin{equation*}
	f(x) = x \cdot f_1(x^2) + f_2(x^2), f_1(x) = 2x, f_2(x) = 3x + 1
\end{equation*}
\begin{equation*}
	g(x) = x \cdot g_1(x^2) + g_2(x^2), g_1(x) = 1, g_2(x) = 2x 
\end{equation*}

Здесь мы переводим многочлен в точки, перемножаем его и обратно переводим его.
\begin{table}[H]
	\begin{tabular}{|l|l|l|l|l|l|l|l|}
	\hline
	x  & $f_1(x)$ & $f_2(x)$ & f(x) & $g_1(x)$ & $g_2(x)$ & g(x) & $f(x) \cdot g(x)$ \\ \hline
	1  & 2       & 4       & 6    & 1       & 2       & 3    & 18                             \\ \hline
	-1 & 2       & 4       & 2   & 1       & 2       & 1    & -2                             \\ \hline
	2  & 8       & 13      & 29   & 1       & 8       & 10   & 290                            \\ \hline
	-2 & 8       & 13      & -3   & 1       & 8       & 6    & -18                            \\ \hline
	3  & 18      & 28      & 82   & 1       & 18      & 21   & 1722                           \\ \hline
	-3 & 18      & 28      & -26  & 1       & 18      & 15   & -390                           \\ \hline
	\end{tabular}
\end{table}

Теперь восстановим коэффициенты многочлена по значения в точках.
\begin{equation*}
	A(x) = \displaystyle \sum_{k = 0}^{n-1} y_k \frac{\displaystyle \prod_{j \not = k}(x - x_j)}{\displaystyle \prod_{j \not = k} (x_k - x _j)}	
\end{equation*}

Я правда не знаю в каком месте это БЫСТРОЕ преобразование Фурье.

	$A(x) = 18 	 	\cdot  \frac{(x+1)(x-2)(x+2)(x-3)(x+3)}{(1+1)(1-2)(1+2)(1-3)(1+3)}
		 + 2	 	\cdot  \frac{(x-1)(x-2)(x+2)(x-3)(x+3)}{(-1-1)(-1-2)(-1+2)(-1-3)(-1+3)} \\
		 + 290 	 	\cdot  \frac{(x-1)(x+1)(x+2)(x-3)(x+3)}{(2-1)(2+1)(2+2)(2-3)(2+3)} 
		 + (-18) 	\cdot  \frac{(x-1)(x+1)(x-2)(x-3)(x+3)}{(-2-1)(-2+1)(-2-2)(-2-3)(-2+3)} \\
		 + (1722)	\cdot  \frac{(x-1)(x+1)(x-2)(x+2)(x+3)}{(3-1)(3+1)(3-2)(3+2)(3+3)}
		 + (-390)	\cdot	\frac{(x-1)(x+1)(x-2)(x+2)(x-3)}{(-3-1)(-3+1)(-3-2)(-3+2)(-3-3)}$

\begin{center} МАГИЯ! \end{center}


$A(x) = 4x^5 + 8x^4 + 3x^3+2x^2+x$

\underline{\textbf{Ответ: }} $A(x) = 4x^5 + 8x^4 + 3x^3+2x^2+x$
\end{document}