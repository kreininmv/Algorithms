\documentclass[a4paper,14pt]{article} % тип документа
%\documentclass[14pt]{extreport}
\usepackage{extsizes} % Возможность сделать 14-й шрифт


\usepackage{geometry} % Простой способ задавать поля
\geometry{top=25mm}
\geometry{bottom=35mm}
\geometry{left=20mm}
\geometry{right=20mm}

\setcounter{section}{0}

%%%Библиотеки
%\usepackage[warn]{mathtext}
%\usepackage[T2A]{fontenc} % кодировка
\usepackage[utf8]{inputenc} % кодировка исходного текста
\usepackage[english,russian]{babel} % локализация и переносы
\usepackage{caption}
\usepackage{listings}
\usepackage{amsmath,amsfonts,amssymb,amsthm,mathtools}
\usepackage{wasysym}
\usepackage{graphicx}%Вставка картинок правильная
\usepackage{float}%"Плавающие" картинки
\usepackage{wrapfig}%Обтекание фигур (таблиц, картинок и прочего)
\usepackage{fancyhdr} %загрузим пакет
\usepackage{lscape}
\usepackage{xcolor}
\usepackage{dsfont}
%\usepackage{indentfirst}
\usepackage[normalem]{ulem}
\usepackage{hyperref}




%%% DRAGON STUFF
\usepackage{scalerel}
\usepackage{mathtools}

\DeclareMathOperator*{\myint}{\ThisStyle{\rotatebox{25}{$\SavedStyle\!\int\!\!\!$}}}

\DeclareMathOperator*{\myoint}{\ThisStyle{\rotatebox{25}{$\SavedStyle\!\oint\!\!\!$}}}

\usepackage{scalerel}
\usepackage{graphicx}
%%% END 

%%%Конец библиотек

%%%Настройка ссылок
\hypersetup
{
colorlinks=true,
linkcolor=blue,
filecolor=magenta,
urlcolor=blue
}
%%%Конец настройки ссылок


%%%Настройка колонтитулы
	\pagestyle{fancy}
	\fancyhead{}
	\fancyhead[L]{Домашнее задание}
	\fancyhead[R]{Крейнин Матвей, группа Б05-005}
	\fancyfoot{}
    \fancyfoot[C]{\thepage}
    \fancyfoot[R]{Основные алгоритмы}
%%%конец настройки колонтитулы



\begin{document}
%%%%Начало документа%%%%

\section{Задание 5}

\subsection{Задача 1}
Задачу можно свести к тому, где находится чило 24, с вероятностью $\frac{1}{2}$ она находится в первой
перестановке, и с вероятностью $\frac{1}{2}$ во второй перестановке.

\underline{\textbf{Ответ: }} $\mathds{P} = \frac{1}{2}$

\subsection{Задача 2}
\begin{equation*}
	\mathds{P}(x \text{ делится на 2}| \text{x делится на 3}) = \frac{\mathds{P}(x \text{ делится на 6})}{\mathds{P}(\text{x делится на 3})}	
\end{equation*}
\begin{equation*}
	\frac{\mathds{P}(x \text{ делится на 6})}{\mathds{P}(\text{x делится на 3})} = \frac{16/100}{33/100} = \frac{16}{33}
\end{equation*}

\underline{\textbf{Ответ: }} $\mathds{P}(x \text{делится на 2}| \text{x делится на 3}) = \frac{16}{33}$

\subsection{Задача 3}
$\mathds{P}(A \cap B) = \mathds{P}(A) \cdot \mathds{P}(B)$ -- одно из определений независимости событий
\newline
A -- среди выбранных чисел есть 2, B -- среди выбранных чисел есть 3.
\newline
\begin{equation*}
	\mathds{P}(A) = \frac{C_{35}^4}{C_{36}^5}; \mathds{P}(B) = \frac{C_{35}^4}{C_{36}^5}	
\end{equation*}
\begin{equation*}
	\mathds{P}(A \cap B) = \frac{C_{34}^3}{C_{36}^5}	
\end{equation*}
\begin{equation*}
	\mathds{P}(B) \cdot \mathds{P}(A) = \frac{(C_{35}^4)^2}{(C_{36}^5)^2} \not = \mathds{P}(A \cap B) = \frac{C_{34}^3}{C_{36}^5}	
\end{equation*}

\underline{\textbf{Ответ: }} Нет.

\subsection{Задача 4}
A -- <<f инъективна>>, B -- <<f(1) $= 1$>>.
\begin{equation*}
	\mathds{P}(A) = \frac{n!}{n^n}; \mathds{P}(B) = \frac{1/n \cdot (n-1)^{n-1}}{n^n}	
\end{equation*}

\begin{equation*}
	\mathds{P}(A \cup B) = \frac{(n-1)!}{n^n}
\end{equation*}

\begin{equation*}
	\mathds{P}(A) \cdot \mathds{P}(B) = \frac{(n-1)! \cdot (n-1)^{n-1}}{n^{2n}} \not = \mathds{P}(A \cup B) = \frac{(n-1)!}{n^n}
\end{equation*}
Т.к. $(n-1)^{n-1} \not = n^n$

\underline{\textbf{Ответ: }} Нет.

\subsection{Задача 5}
\begin{equation*}
	\mathds{P} = p * p * 1 + p * (1-p) * 1/2 + (1-p) * p * 1/2 = p^2 - p^2 + p = p
\end{equation*}

\begin{equation*}
	\mathds{P} = p \text{  или  } \mathds{P} = 1/2
\end{equation*}

\underline{\textbf{Ответ:}} $\mathds{P} = p$, зависит от того больше ли $p > 1/2$ или меньше.

\subsection{Задача 6}
Пусть X шариков в первой коробке и x из них белые, Y - во второй корбке и y из них белые. $x + y = 10, X + Y = 20, X \not = 0, Y \not = 0, x \leq X, y \leq Y$
\newline
Получаем, что $x \leq X$ and $10 - x \leq 20 - X$; $0 \leq X - x \leq 10$
\begin{equation*}
	\mathds{P} = \frac{1}{2} \cdot (\frac{x}{X} + \frac{10-x}{20-X}) = \frac{1}{2} \cdot \frac{(20x - Xx + 10X - xX)}{X (20 - X)}
\end{equation*}
\begin{equation*}
	\mathds{P} = \frac{1}{2} \cdot \frac{(20x - Xx + 10X - xX)}{X (20 - X)} 
\end{equation*}
Посмотрели на вторые производные и увидели, что экстремума нет, поэтому смотрим на граничные условия.
Из них понимаем, что максимум будет при $X = 1, x = 1$.

\underline{\textbf{Ответ:}} $P = \frac{14}{19}$

\subsection{Задача 7}
Составим табличку:
\begin{table}[H]
	\begin{tabular}{|l|l|l|l|l|}
	\hline
	10 & 1     		& 1   & 1   &    \\ \hline
	9  & $7/8  $ 	& $3 /4$ & $1/2$ & 0  \\ \hline
	8  & $11/16$ 	& $1 /2$ & $1/4$ & 0  \\ \hline
	   & 7     		& 8   & 9   & 10 \\ \hline
	\end{tabular}
\end{table}
В ячейке указана вероятность выиграть первому.
Таблица формируется следующим образом: берется вероятность <<выиграть>> из правой клетки и умножается на вероятность перейти в эту клетку и к ней суммируется произведение вероятности выиграть из верхней клетки и вероятности попасть в неё.
Заполнять таблицу начинаем из правого верхнего угла. Вероятность выиграть, когда первый игрок уже выиграл 10 партий равна 1. Вероятность, когда второй игрок выиграл 10 партий равна нулю.

\underline{\textbf{Ответ: }} $\mathds{P} = \frac{11}{16}$

\subsection{Задача 8}
У нас $n + 2$ яйца, они все различаются по прочности. Очевидно, чтобы он выиграл нужно, чтобы у первого игрока было самое крепкое яйцо, т.е. стояло бы на первом месте, а все остальные неважно как стоят, т.е. $(n+1)!$ вариант.
Всего же вариантов: $n! + (n+1)!$, т.к. яйцо у первого игрока может стоять либо на втором месте, либо на первом. Если оно стоит на втором месте, то вариантов $n!$, если на первом, то $(n+1)!$.

$\frac{(n+1)!}{n! + (n+1)!} = \frac{n+1}{n+2}$

\underline{\textbf{Ответ: }} $\frac{n+1}{n+2}$.

\subsection{Задача 9}
Посмотрим на позиции с 1 по 10 (x -- количество единиц в этой части), и с 11 по 20 (y -- количество единиц в этой части). 
Вероятность того, что $x < y$ такая же, как и $y < x$, $p_1 = p_2$
В первом случае, что у нас будет стоять на 21-м месте не имеет значение уже единиц больше. Во втором случае тоже не будет разницы, т.к. в лучшем случае их количество будет равно.
Осталось рассмотреть случай, когда $x = y$, у этого какая-то вероятность $p$. Тогда с вероятностью $1/2 \cdot p$ их будет больше, и с вероятностью $1/2 \cdot p$ их будет меньше.

Т.е. получаем вероятность $\mathds{P} = p_1 + \frac{1}{2} p_3$, но $p_1 + p_2 + p_3 = 1$ and $p_1 = p_2$, 
\newline
т.е. $\mathds{P} = \frac{1}{2}$

\underline{\textbf{Ответ: }} $\mathds{P} = \frac{1}{2}$


\end{document}