\documentclass[a4paper,14pt]{article} % тип документа
%\documentclass[14pt]{extreport}
\usepackage{extsizes} % Возможность сделать 14-й шрифт


\usepackage{geometry} % Простой способ задавать поля
\geometry{top=25mm}
\geometry{bottom=35mm}
\geometry{left=20mm}
\geometry{right=20mm}

\setcounter{section}{0}

%%%Библиотеки
%\usepackage[warn]{mathtext}
%\usepackage[T2A]{fontenc} % кодировка
\usepackage[utf8]{inputenc} % кодировка исходного текста
\usepackage[english,russian]{babel} % локализация и переносы
\usepackage{caption}
\usepackage{listings}
\usepackage{amsmath,amsfonts,amssymb,amsthm,mathtools}
\usepackage{wasysym}
\usepackage{graphicx}%Вставка картинок правильная
\usepackage{float}%"Плавающие" картинки
\usepackage{wrapfig}%Обтекание фигур (таблиц, картинок и прочего)
\usepackage{fancyhdr} %загрузим пакет
\usepackage{lscape}
\usepackage{xcolor}
\usepackage{dsfont}
%\usepackage{indentfirst}
\usepackage[normalem]{ulem}
\usepackage{hyperref}




%%% DRAGON STUFF
\usepackage{scalerel}
\usepackage{mathtools}

\DeclareMathOperator*{\myint}{\ThisStyle{\rotatebox{25}{$\SavedStyle\!\int\!\!\!$}}}

\DeclareMathOperator*{\myoint}{\ThisStyle{\rotatebox{25}{$\SavedStyle\!\oint\!\!\!$}}}

\usepackage{scalerel}
\usepackage{graphicx}
%%% END 

%%%Конец библиотек

%%%Настройка ссылок
\hypersetup
{
colorlinks=true,
linkcolor=blue,
filecolor=magenta,
urlcolor=blue
}
%%%Конец настройки ссылок


%%%Настройка колонтитулы
	\pagestyle{fancy}
	\fancyhead{}
	\fancyhead[L]{Домашнее задание}
	\fancyhead[R]{Крейнин Матвей, группа Б05-005}
	\fancyfoot{}
    \fancyfoot[C]{\thepage}
    \fancyfoot[R]{Основные алгоритмы}
%%%конец настройки колонтитулы



\begin{document}
%%%%Начало документа%%%%

\section{Задание 5}

\subsection{Задача 1}
Задачу можно свести к тому, где находится чило 24, с вероятностью $\frac{1}{2}$ она находится в первой
перестановке, и с вероятностью $\frac{1}{2}$ во второй перестановке.

\underline{\textbf{Ответ: }} $\mathds{P} = \frac{1}{2}$

\subsection{Задача 2}
\begin{equation*}
	\mathds{P}(x \text{делится на 2}| \text{x делится 3}) = \frac{\mathds{P}(x \text{ делится на 6})}{\mathds{P}(\text{x делится 3})}	
\end{equation*}
\begin{equation*}
	\frac{\mathds{P}(x \text{ делится на 6})}{\mathds{P}(\text{x делится 3})} = \frac{16/100}{33/100} = \frac{16}{33}
\end{equation*}

\underline{\textbf{Ответ: }} $\mathds{P}(x \text{делится на 2}| \text{x делится 3}) = \frac{16}{33}$

\subsection{Задача 3}
$\mathds{P}(A \cap B) = \mathds{P}(A) \cdot \mathds{P}(B)$ -- одно из определений независимости событий
\newline
A -- среди выбранных чисел есть 2, B -- среди выбранных чисел есть 3.
\newline
\begin{equation*}
	\mathds{P}(A) = \frac{C_{35}^4}{C_{36}^5}; \mathds{P}(B) = \frac{C_{35}^4}{C_{36}^5}	
\end{equation*}
\begin{equation*}
	\mathds{P}(A \cap B) = \frac{C_{34}^3}{C_{36}^5}	
\end{equation*}
\begin{equation*}
	\mathds{P}(B) \cdot \mathds{P}(A) = \frac{(C_{35}^4)^2}{(C_{36}^5)^2} \not = \mathds{P}(A \cap B) = \frac{C_{34}^3}{C_{36}^5}	
\end{equation*}

\underline{\textbf{Ответ: }} Нет.

\subsection{Задача 4}
A -- <<f инъективна>>, B -- <<f(1) = 1>>.
\begin{equation*}
	\mathds{P}(A) = \frac{n!}{n^n}; \mathds{P}(B) = \frac{1/n \cdot (n-1)^{n-1}}{n^n}	
\end{equation*}

\begin{equation*}
	\mathds{P}(A \cup B) = \frac{(n-1)!}{n^n}
\end{equation*}

\begin{equation*}
	\mathds{P}(A) \cdot \mathds{P}(B) = \frac{(n-1)! \cdot (n-1)^{n-1}}{n^{2n}} \not = \mathds{P}(A \cup B) = \frac{(n-1)!}{n^n}
\end{equation*}
Т.к. $(n-1)^{n-1} \not = n^n$

\underline{\textbf{Ответ: }} Нет.

\subsection{Задача 5}
\begin{equation*}
	\mathds{P} = p * p * 1 + p * (1-p) * 1/2 + (1-p) * p * 1/2 = p^2 - p^2 + p = p
\end{equation*}

\begin{equation*}
	\mathds{P} = p \text{  или  } \mathds{P} = 1/2
\end{equation*}

\underline{\textbf{Ответ:}} $\mathds{P} = p$, зависит от того больше ли $p > 1/2$ или меньше.


\end{document}